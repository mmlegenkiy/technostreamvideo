%\Xi 
\documentclass[t]{beamer}  % [t], [c], или [b] --- вертикальное выравнивание на слайдах (верх, центр, низ)
%\documentclass[handout]{beamer} % Раздаточный материал (на слайдах всё сразу)
%\documentclass[aspectratio=169]{beamer} % Соотношение сторон

%\usetheme{Berkeley} % Тема оформления
%\usetheme{Bergen}
%\usetheme{Szeged}

%\usecolortheme{beaver} % Цветовая схема
%\useinnertheme{circles}
%\useinnertheme{rectangles}

%\usetheme{HSE}

%%% Работа с русским языком
\usepackage{cmap}					% поиск в PDF
\usepackage{mathtext} 				% русские буквы в формулах
\usepackage[T2A]{fontenc}			% кодировка
\usepackage[utf8]{inputenc}			% кодировка исходного текста
\usepackage[english,ukrainian]{babel}	% локализация и переносы

\usepackage{hyperref}

%%% Работа с картинками
\usepackage{graphicx}  % Для вставки рисунков
\graphicspath{{images/}{images2/}}  % папки с картинками
\setlength\fboxsep{3pt} % Отступ рамки \fbox{} от рисунка
\setlength\fboxrule{1pt} % Толщина линий рамки \fbox{}
\usepackage{wrapfig} % Обтекание рисунков текстом

%%% Работа с таблицами
\usepackage{array,tabularx,tabulary,booktabs} % Дополнительная работа с таблицами
\usepackage{longtable}  % Длинные таблицы
\usepackage{multirow} % Слияние строк в таблице

%%% Программирование
\usepackage{etoolbox} % логические операторы

%%% Другие пакеты
\usepackage{lastpage} % Узнать, сколько всего страниц в документе.
\usepackage{soul} % Модификаторы начертания
\usepackage{csquotes} % Еще инструменты для ссылок
%\usepackage[style=authoryear,maxcitenames=2,backend=biber,sorting=nty]{biblatex}
\usepackage{multicol} % Несколько колонок

%%% Картинки
\usepackage{tikz} % Работа с графикой
\usepackage{pgfplots}
\usepackage{pgfplotstable}

\title{LINUX Технострім}
%\subtitle{За матеріалами "Системне адміністрування Linux" Сергія Клочкова}
\author{Легенький М.М.}
\date{\today}
\institute[факультет радіофізики, біомедичної електроніки та комп'ютерних систем]{Харківський національний університет імені В. Н. Каразіна}

\begin{document}

\frame[plain]{\titlepage}	% Титульный слайд

\section{Робота з командним рядком}
\subsection{Вступ}
 
\begin{frame}
	\frametitle{\insertsection} 
	\framesubtitle{\insertsubsection}
Дистрибутиви Linux.

Способи поширення ПЗ: у вигляді вихідних кодів (Gentoo) або у бінарному вигляді (Ubuntu). Політика оновлення ПЗ: стабільність (RHEL) або актуальність (Archlinux).		 

Єдине дерево директорій.
\begin{itemize}
\item /etc - конфігурація.
\item /home - дані користувачів.
\item /usr - додатки.
\item /var - дані та логи.
\item /tmp - тимчасові файли.
\end{itemize}
\end{frame}

\subsection{Основи роботи в системі Linux}
 
\begin{frame}
	\frametitle{\insertsection} 
	\framesubtitle{\insertsubsection}
Порядок завантаження персонального комп'ютера системи IBM PC:
\begin{itemize}
\item Ініціалізація та перевірка працезданості обладнання (POST).
\item Передача керування завантажувачу з якогось локального пристрою  або мережею.
\item Для завантаження з локального диску необхідно, щоб на цьому диску була таблиця розділів, що підтримується BIOS, та був завантажувач.
\item Основні види таблиць розділів MBR (sudo dd if=/dev/sda bs=512 count=1 $|$ file -) та GPT (використовується службовий розділ розміром 2 Мб, де знаходиться завантажувач). 
\item Завантажувач з MBR (stage1) запускає інший більш фунціональний завантажувач (stage2). 
\item Далі завантажувачу треба запустити ядро операційної системи (при цьому може демонструватись меню).
\item initrd використовується щоб безпосередньо розпочати роботу з обладнаннням.
\item Далі запускається init щоб запустити всі процеси та примонтувати всі файлові систем окрім кореневої.
\end{itemize}

\end{frame}

\begin{frame}
	\frametitle{\insertsection} 
	\framesubtitle{\insertsubsection}
Ядро ОС потрібне для абстракції обладнання, для керування виділенням ресурсів користувацьким процесам  та для забезпечення доступу процесів до деякої низки ресурсів (за допомогою системних викликів).

Команда sysctl дозволяє продивлятись та змінювати параметри ядра Linux.

Утиліта modprobe використовується для додавання завантажуваних модулів до ядря ​​Linux. Можна продивлятися та видаляти модулі за допомогою команди modprobe.
\end{frame}

\end{document}
