%\Xi 
\documentclass[t]{beamer}  % [t], [c], или [b] --- вертикальное выравнивание на слайдах (верх, центр, низ)
%\documentclass[handout]{beamer} % Раздаточный материал (на слайдах всё сразу)
%\documentclass[aspectratio=169]{beamer} % Соотношение сторон

%\usetheme{Berkeley} % Тема оформления
%\usetheme{Bergen}
%\usetheme{Szeged}

%\usecolortheme{beaver} % Цветовая схема
%\useinnertheme{circles}
%\useinnertheme{rectangles}

%\usetheme{HSE}

%%% Работа с русским языком
\usepackage{cmap}					% поиск в PDF
\usepackage{mathtext} 				% русские буквы в формулах
\usepackage[T2A]{fontenc}			% кодировка
\usepackage[utf8]{inputenc}			% кодировка исходного текста
\usepackage[english,ukrainian]{babel}	% локализация и переносы

\usepackage{hyperref}

%%% Работа с картинками
\usepackage{graphicx}  % Для вставки рисунков
\graphicspath{{images/}{images2/}}  % папки с картинками
\setlength\fboxsep{3pt} % Отступ рамки \fbox{} от рисунка
\setlength\fboxrule{1pt} % Толщина линий рамки \fbox{}
\usepackage{wrapfig} % Обтекание рисунков текстом

%%% Работа с таблицами
\usepackage{array,tabularx,tabulary,booktabs} % Дополнительная работа с таблицами
\usepackage{longtable}  % Длинные таблицы
\usepackage{multirow} % Слияние строк в таблице

%%% Программирование
\usepackage{etoolbox} % логические операторы

%%% Другие пакеты
\usepackage{lastpage} % Узнать, сколько всего страниц в документе.
\usepackage{soul} % Модификаторы начертания
\usepackage{csquotes} % Еще инструменты для ссылок
%\usepackage[style=authoryear,maxcitenames=2,backend=biber,sorting=nty]{biblatex}
\usepackage{multicol} % Несколько колонок

%%% Картинки
\usepackage{tikz} % Работа с графикой
\usepackage{pgfplots}
\usepackage{pgfplotstable}

\title{LINUX Технострім}
%\subtitle{За матеріалами "Системне адміністрування Linux" Сергія Клочкова}
\author{Легенький М.М.}
\date{\today}
\institute[факультет радіофізики, біомедичної електроніки та комп'ютерних систем]{Харківський національний університет імені В. Н. Каразіна}

\begin{document}

\frame[plain]{\titlepage}	% Титульный слайд

\section{Робота з командним рядком}
\subsection{Вступ}
 
\begin{frame}
	\frametitle{\insertsection} 
	\framesubtitle{\insertsubsection}
Дистрибутиви Linux.

Способи поширення ПЗ: у вигляді вихідних кодів (Gentoo) або у бінарному вигляді (Ubuntu). Політика оновлення ПЗ: стабільність (RHEL) або актуальність (Archlinux).		 

Єдине дерево директорій.
\begin{itemize}
\item /etc - конфігурація.
\item /home - дані користувачів.
\item /usr - додатки.
\item /var - дані та логи.
\item /tmp - тимчасові файли.
\end{itemize}
\end{frame}

\subsection{Основи роботи в системі Linux}
 
\begin{frame}
	\frametitle{\insertsection} 
	\framesubtitle{\insertsubsection}
Порядок завантаження персонального комп'ютера системи IBM PC:
\begin{itemize}
\item Ініціалізація та перевірка працезданості обладнання (POST).
\item Передача керування завантажувачу з якогось локального пристрою  або мережею.
\item Для завантаження з локального диску необхідно, щоб на цьому диску була таблиця розділів, що підтримується BIOS, та був завантажувач.
\item Основні види таблиць розділів MBR (sudo dd if=/dev/sda bs=512 count=1 $|$ file -) та GPT (використовується службовий розділ розміром 2 Мб, де знаходиться завантажувач). 
\item Завантажувач з MBR (stage1) запускає інший більш фунціональний завантажувач (stage2). 
\item Далі завантажувачу треба запустити ядро операційної системи (при цьому може демонструватись меню).
\item initrd використовується щоб безпосередньо розпочати роботу з обладнаннням.
\item Далі запускається init щоб запустити всі процеси та примонтувати всі файлові систем окрім кореневої.
\end{itemize}

\end{frame}

 \begin{frame}
 	\frametitle{\insertsection} 
 	\framesubtitle{\insertsubsection}
Ядро ОС потрібне для абстракції обладнання, для керування виділенням ресурсів користувацьким процесам  та для забезпечення доступу процесів до деякої низки ресурсів (за допомогою системних викликів).

Команда sysctl дозволяє продивлятись та змінювати параметри ядра Linux.

Утиліта modprobe використовується для додавання завантажуваних модулів до ядра Linux. Можна продивлятися та видаляти модулі за допомогою команди modprobe.
\end{frame}

\begin{frame}
 	\frametitle{\insertsection} 
 	\framesubtitle{\insertsubsection}
Операційна система розв'язує різні задачі за допомогою керування процесами. Процес складається із образу коду, що виконується, адресного простору, набору дескрипторів (нумеровані ресурси, що виділені процесу ядром, вони відповідають за операції вводу-виводу), атрибути доступу, контекст процесора (регістри, вказівники на стеки тощо).

Псевдофайлова система procfs примонтована в директорію /proc. Інформацію про процес можна подивитись у файлі /proc/PID/status.

Виділений процесу адресний простір не завжди відповідає реально зайнятій пам'яті. Реально виділяється пам'ять тільки, коли процес намагається записати туди дані (це робиться за допомогою таблиці трансляції сторінок). Якими блоками виділяється пам'ять? 4 Кб - сторінка пам'яті в архітектурі х86, адреса сторінки 8 байт. 
\end{frame}

\begin{frame}
 	\frametitle{\insertsection} 
 	\framesubtitle{\insertsubsection}
Робота із таблицею трансляції сторінок є критичною і вона реалізована апаратно. За неї відповідає MMU - модуль керування пам'яттю в центральному процесорі. Також для цього виділено дуже швидкий та дуже маленький кеш (на багатьох процесорах приблизно 1024 записи). Однак 1024 * 4 Кб = 4 Мб. Якщо ж відбувається звертання до сторінки, що не знаходиться в цьому кеші, доводиться звертатися до оперативної пам'яті і це довго. 

Збільшення розміру сторінки завжди призводе до збільшення кількості оперативної пам'яті, що не використовується.

\end{frame}

\begin{frame}
 	\frametitle{\insertsection} 
 	\framesubtitle{\insertsubsection}
Тому алокацію пам'яті в архитектурі х64 можна зробити не тільки на сторінку розміром 4 Кб, але і на велику сторінку (Huge page) розміром 2 Мб. Таким чином, в швидкому кеші знаходяться 1024* 2Мб = 2 Гб даних.
 За допомогою sysctl можна виділити кількість великих сторінок, які будуть в системі sysctl vm.nr\_hugepages. До цієї пам'яті можуть звернутися лише ті додатки, для яких її призначено. Ці звертання завжди будуть швидкими (пам'ять не може бути витіснена в режим підкачки). Недолік: мінімальна гибкість. Виділення сторінок повинно відбуватися при старті системи, вони повинні виділятися підряд.
 
 Тому для роботи з великими сторінками придумали прозорі великі (анонімні) великі сторінки. cat /proc/meminfo $|$ grep AnonHugePages. Ядро саме вирішує які саме сторінки пам'яті виділити додатку, але ці сторінки можуть бути витіснені до розділу підкачки.
\end{frame}

\begin{frame}
 	\frametitle{\insertsection} 
 	\framesubtitle{\insertsubsection}
Інший вид ресурсу, що виділяється процесу - дескриптор вводу/виводу. Стандартні дескриптори - 0 (stdin), 1 (stdout) та 2 (stderr). Список виділених процесу дескрипторів знаходиться в директорії /proc/PID/fd. Інформацію про ці дескриптори можна подивитись у файлі /proc/PID/fdinfo/PID1 (опції, з якими їх було відкрито, позиції; де знаходиться в файлі процес тощо).

lsof - утиліта, що виводе інформацію про те які файли використовуються тими чи іншими процесами. lsof в стовпці FD виводе різні службові значення: txt (виконуваний файл додатку), mem (різні файли, розміщені в адресному просторі даного процесу), cwd (поточна робоча директорія), rtd (коренева директорія процесу).
\end{frame}

\begin{frame}
 	\frametitle{\insertsection} 
 	\framesubtitle{\insertsubsection}
Головний ресурс процесу - ресурси процесора. Процес використовує час  процесору в коді додатку (us (user space) в top) та в коді ядра (sy в top). Для моніторингу використовується утиліта top (ni -процеси зі зменшеним пріорітетом, wa - очикування вводу/виводу, hi - апаратні переривання , si - програмні переривання, st - час, витрачений іншими віртуальними машинами ).

iostat - утиліта, призначена для моніторингу використання дискових розділів. Команда dd if=/dev/sda of=/dev/null bs=1M

Системні виклики - механізм взаємодії ядра та додатків. При системному виклику додаток у своєму контексті запускає визначений код ядра. Системні виклики потрібні для операцій вводу та виводу або для того, щоб надсилати команди системі, за допомогою системних викликів запрашується виділення ресурсів.
\end{frame}

\begin{frame}
 	\frametitle{\insertsection} 
 	\framesubtitle{\insertsubsection}
strace - утиліта, яка дозволяє відслідкувати виконання системних викликів та сигналів до ядра системи.

Як можна взаємодіяти з додатком? Сигнали - механізм зв'язку з додатком. Інший спосіб як система може нагадати процесу про себе - це ліміти. Ліміти виписуються на користувача і розділюються між процесами даного користувача. Команда ulimit дозволяє відображати та встановлювати обмеження ресурсів для користувачів. Для керування лімітами існує файл /etc/security/limits.conf . Існують м'які (soft) та жорсткі (hard) ліміти. М'який ліміт - ліміт за замовчуванням при вході в систему, жорсткий ліміт - межа, до якої користувач може підняти свої ліміти. 
\end{frame}

\begin{frame}
 	\frametitle{\insertsection} 
 	\framesubtitle{\insertsubsection}
Змінні оболонки - низка ключів та значень, що відноситься до процесу та змінюється динамічно. 

Самий перший процес в оточенні користувача - init запускає ядро при завантаженні системи. В UNIX-системі є лише один спосіб породження процесу - клонування самого себе, для цього використовується функція fork()  (раніше використовувався системний виклик fork()).

gdb - основний відлагоджувач, що використовується у багатьох UNIX-системах.
\end{frame}

\subsection{Linux та мережа}

\begin{frame}
 	\frametitle{\insertsection} 
 	\framesubtitle{\insertsubsection}
Мультикаст дозволяє звернутися до серверу та підписатися на отримання інформації так, щоб здійснити мінімальне навантаження на сервер. Якщо у сервера є тільки один маршрут до провайдера, то всі пакети туди будуть йти лише в одному екземплярі. А далі вони будуть  надсилатися лише на ті маршрутизатори, за якими є хоча б один клієнт, що підписався на мультикаст. Щоб мультикаст працював виділено спеціальну підмережу 224.0.0.0/4. Якщо клієнт підписується на отримання мультикастного трафіка, він повідомяє про це своєму локаьного роутеру по протоколу IGMP. Цей роутер зв'язується з іншими роутерами по протоколу PIM і повідомляє, що має клієнта, який підписався на такий-то мультикаст.

При мультикасті не гарантується порядок пакетів і немає можливості отримати втрачений пакет.
\end{frame}

\begin{frame}
 	\frametitle{\insertsection} 
 	\framesubtitle{\insertsubsection}
Головний недолік IPv4 - довжина адреси ( $2^{32}$ адрес). В IPv6 довжину адреси збільшили в 4 рази ( $2^{128}$ адрес). З'явився новий механізм автоконфігурації адрес, хост сам себе анонсує. Адреси видаються пулами по $2^{64}$ адрес. В IPv6 маршрутизатори можуть сами себе анонсувати. Можна легко підібрати собі унікальну адресу в межах даної мережі. Кожен мережевий інтерфейс має унікальну МАС-адресу довжиною 48 біт. В IPv6 дозволено мати в мережі декілька маршрутизаторів (до трьох). В IPv6 легко перехоплювати трафік, тому важливим є шифрування. Роутери більше не фрагментують пакети.

Для того щоб дізнатися для якого додатку призначено пакет використовується номер порта. В протоколі UDP до IP-адреси додається номер порта.
\end{frame}

\begin{frame}
 	\frametitle{\insertsection} 
 	\framesubtitle{\insertsubsection}
TCP - протокол гарантованої доставки. Для того, щоб ініціювати з'єднання клієнт надсилає пакет з флагом SYN і повідомляє пакети, якого максимального розміру він може обробляти. Сервер також надсилає пакет з флагом SYN та ACK (підтвердження) і повідомляє пакети, якого максимального розміру він може обробляти. Після цього клієнт надсилає серверу ACK, щоб показати, що отримав його SYN та ACK.

Після цього йде обмін пакетами. Клієнт  надсилає наприклад байти до 1024. Сервер відповідає 1025, тобто 1024 отримав нормально, чекаю на 1025-й. Після того, як з'єднання стає непотрібним, його закривають. Для цього будь-яка сторона клієнт або сервер надсилає пакет FIN. У відповідь йде FIN та ACK, тобто друга сторона підтверджує закриття з'єднання і також каже, що вона його закриває. Після цього перша сторона каже ACK і з'єднання вважається повністю закритим. 
\end{frame}

\begin{frame}
 	\frametitle{\insertsection} 
 	\framesubtitle{\insertsubsection}
Для того щоб мережеве обладнання працювало достатньо швидко треба надсилати пакети, не очікуючи підтверджень. Но яку кількість даних слід надсилати? TCP congestion control. Вводиться поняття TCP-вікна: кількість байт, яку можна надсилати, не очікуючи підтвердження. Як правильно підібрати розмір вікна? В кожній мережевій топології він буде різним. Для цього використовується адаптивний розмір вікна. На початку з'єднання він малий, але зростає доки не трапляється таймаут при очікуванні ACK. При таймауті розмір вікна зменшується до 1 mss, ліміт збільшення вікна виставляється в половину розміра вікна до таймаута.
\end{frame}

\begin{frame}
 	\frametitle{\insertsection} 
 	\framesubtitle{\insertsubsection}
Як бути з сірими адресами? Для того щоб забезпечити вам доступ до інтернет провайдер використовує технологію NAT (Network Addres Translation). У провайдера є пул адрес (в простійшому випадку один білий адрес), який використовується як вихідна адреса для ваших з'єднань. В обладнанні провайдера є таблиця, що ваше з'єднання з такої-то сірої адреси і такого-то порта, відповідає з'єднанню, яке провайдер встанове зі своєї білої адреси і такого-то порта. Коли сервер вирішить надіслати вам відповідь він буде використовувати свою адресу та порт як вихідний і той порт, який видано, як порт призначення. Пакет, коли потрапе на обладнання провайдера буде оброблений у відповідності до таблиці NAT, адресу призначення буде змінено на вашу адресу, а порт - на ваш порт, і цей пакет буде передано вам. Отже, всі клієнти зможуть працювати доки сумарна кількість з'єднань не перевище кількість білих адрес провайдера помножену на максимально можливу кількість портів (65536).
\end{frame}

\begin{frame}
 	\frametitle{\insertsection} 
 	\framesubtitle{\insertsubsection}
DNS (Domain Name System) - протокол рівня додатків. Зазвичай користувачі не використовують ір-адреси для підключення до якогось сервісу. Для того щоб разрішати доменні імена в ір-адреси використовується DNS-сервер. Це розподілена ієрархічна база даних, що дозволяє делегувати виділення відповідностей між визначеними іменами визначеним адресам власникам доменів. 

У DNS домен більш верхнього рівня йде з правого боку, а далі йдуть домени більш високого рівня. Точка - кореневий домен, він обслуговується кореневими секторами, адреси яких добре відомі і ніколи не змінюються.  Вони містять дуже важливі відомості, якими серверами обслуговуються домени com, net  і так далі. Після цього ми звертаємося до одного із них і дізнаємося у що резолвиться шуканий сайт. IN NS - позначає делегування.
\end{frame}

\begin{frame}
 	\frametitle{\insertsection} 
 	\framesubtitle{\insertsubsection}
Утиліта dig використовується для роботи з доменними записами, +trace - трасування маршруту запиту.  NS - name server, A - IPv4 адреса, AAAA - IPv6 адреса, SOA (start of authority record) - запис, де зберігаються політики для домену.
\end{frame}

\end{document}
